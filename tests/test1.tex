\documentclass[11pt,english,twoside]{article}
\usepackage{graphicx}

\begin{document}

\title{SPEAD: Streaming Protocol for Exchanging Astronomical Data}
\author{Jason Manley, Marc Welz, Aaron Parsons}

\references{references}

\section{Scope}

under\_line under\_line
This document describes a datastream format suitable for use by radio astronomy instruments. The data stream is distinct from the \emph{KATCP} control protocol which follows a format described in \cite{ffdbectl}. 

\section{Protocol Layers}

The data is transmitted via a network interface. The network interface consists of the following layers, from top to bottom:

\begin{description}

\item[Application and Presentation]: The data format as described in later sections of this document.

\item[Session and Transport]: UDP, noting that the receiving party can not request retransmission of lost, duplicated or corrupt packets.

\item[Network]: IP (Internet protocol)

\item[Link]: 10 Gbps, 1 Gbps or 100 Mbps Ethernet

\item[Physical]: SFP+, CX4 or other 10 Gigabit ethernet cable, Cat6, Cat5e etc.

\end{description}

\section{Overview}

Note that lower speed and ad-hoc data products may be transferred using the \emph{KATCP} control interface \cite{ffdbectl}.

\subsection{Header}

All data stream packets have a header of the following form:
\begin{table}
\begin{tabular}{|r|l|}
\hline
\emph{Number of bits} & \emph{Description} \\
\hline
(MSB) 16 & Magic Number (constant 0x4b52, in ascii \emph{KR}) \\
      16 & Version identifier) \\
      16 & Reserved \\
(LSB) 16 & Number of options \\
\hline
\end{tabular}
\end{table}

\begin{description}

\item[Magic number]:
This is a constant value. All packets will start with the following 16 bit pattern: \emph{(msb) 0100 1011 0101 0010 (lsb)}.

\item[Version identifier]:
This field identifies the version of the packet format. Designed in case future packet structures differ. The version described here is 0003.

\item[Reserved]:
This field should be ignored. The space is reserved for future allocation.

\item[Number of options]:
The value stored in this field indicates the number of optional 64-bit fields following the header. Any attached data follows directly after this number of 64-bit fields.

\end{description}

\subsection{\emph{Options}: Optional Fields}

The following table defines these standardised \emph{options} identifier codes and their meaning.
\begin{table}
\begin{tabular}{|l|l|l|}
\hline
\emph{Dec} & \emph{Hex} & \emph{Description} \\
\hline
 0 & 0000 & NULL - Ignore \\
 1 & 0001 & Instrument type \\
 4 & 0004 & Data payload length \\
 5 & 0005 & Data payload offset \\
13 & 000D & Stream control \\
14 & 000E & Metadata packet counter \\
48 & 0030 & Option descriptor \\
49 & 0031 & Payload descriptor \\
\hline
\end{tabular}
\end{table}

\end{document}
